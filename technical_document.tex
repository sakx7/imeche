\documentclass{article}
\usepackage{graphicx} 
\usepackage{float}
\usepackage{booktabs}
\usepackage{array}
\usepackage{arydshln}
\usepackage{siunitx}
\usepackage{hyperref}
\usepackage{cancel}
\usepackage{changepage}
\usepackage{placeins}
\usepackage{enumitem}
\usepackage{siunitx}
\usepackage{lipsum}
\usepackage[most]{tcolorbox}
\definecolor{light-gray}{gray}{0.9}
\newcommand{\code}[1]{\colorbox{light-gray}{\texttt{#1}}}

%\usepackage{showframe}
\usepackage{times}

\DeclareSIUnit{\atm}{atm}

\usepackage{tabularx}
\usepackage{amsmath, amssymb, amscd, MnSymbol, mathrsfs}
\usepackage{cellspace}
\usepackage{tikz}
\usetikzlibrary{calc,3d, patterns, angles, quotes, decorations.markings, decorations.pathmorphing, hobby}
\usepackage{xfrac}

\usepackage{chemfig}
\usepackage{caption}
\usepackage{bm}
\usepackage{pdfpages}
\usepackage{empheq}
\usepackage{pgfplots}
\usepackage{pgfplotstable}
\usepackage{xstring}

\pgfplotsset{compat=1.18}
\usepackage[oldvoltagedirection]{circuitikz}
\usepackage{microtype}
\usepackage{tikz-3dplot}
\usepackage{textcomp}
% Custom commands
\newcommand{\vect}[1]{\boldsymbol{\mathbf{#1}}}
\newcolumntype{C}{>{\centering\arraybackslash}X}
\newcolumntype{M}[1]{>{\centering\arraybackslash}m{#1}}

%\usetikzlibrary{external}
%\tikzexternalize[prefix=figures/]

\newcommand\myfrac[2]{\sfrac{#1\mkern-1.2mu}{#2}}
\usepackage{xcolor}

% Define custom colors
\definecolor{darkblue}{rgb}{0.1,0.1,0.5} %  dark blue shade
\definecolor{formalshade}{rgb}{0.95,0.95,1} % light blue shade for the background

% For the adjustwidth environment
\PassOptionsToPackage{strict}{changepage}
\usepackage{changepage}

% For formal definitions
\usepackage{framed}

\newcommand{\formalsource}{} % Initialize an empty macro to store the source text

\newenvironment{formal}[3][]{% Start of the environment
	\renewcommand{\formalsource}{#1}% Store the optional argument
	\def\FrameCommand{%
		\hspace{1pt}%
		{\color{#2}\vrule width 2pt}%
		{\color{#3}\vrule width 4pt}%
		\colorbox{#3}%
	}%
	\MakeFramed{\advance\hsize-\width\FrameRestore}%
	\noindent\hspace{-4.55pt}% Disable indenting the first paragraph
	\begin{adjustwidth}{}{7pt}%
		\vspace{2pt}%
	}%
	{%
		\vspace{4pt}%
		\ifx\formalsource\empty % Check if the source is empty
		\else
		\hfill{\footnotesize{\formalsource}}% Align source to the bottom-right
		\fi
	\end{adjustwidth}\endMakeFramed%
}


% Custom itemize list with images for positive and negative items
\newlist{gitemize}{itemize}{1} % Just one level for the list
\setlist[gitemize,1]{
	leftmargin=2.8em, % Adjust the margin for the list
	labelsep=1em % Control the space between the label and the list item
}

% Define checkmark and cross symbols for positive and negative items
\newcommand{\checkitem}{\raisebox{-0.25\height}{\includegraphics[width=0.4cm]{checkmark.png}}}
\newcommand{\crossitem}{\raisebox{-0.25\height}{\includegraphics[width=0.4cm]{cross.png}}}


\usepackage[left=0.8in,right=0.8in,top=0.5in,bottom=0.69in,includeheadfoot,letterpaper]{geometry}
\usepackage{fancyhdr}
\usepackage{graphicx}
\usepackage{tabularray}
\usepackage{varwidth} 


\newcommand{\wm}[2]{%
	\begin{minipage}{#1\textwidth}
		\centering
		#2
	\end{minipage}%
}

\pagestyle{fancy}
\fancyhf{}


\renewcommand{\headrulewidth}{0.4pt}
\renewcommand{\footrulewidth}{0.4pt}

\fancyhead[L]{\includegraphics[height=1.2cm]{images/Kingston_University_London_logo_200-tablet.png}}
\fancyhead[R]{EG4023 – ME – IMechE Design Challenge Project}
\fancyfoot[C]{Department of Mechanical Engineering}
\fancyfoot[R]{\thepage}

\usepackage{scalerel}
\usepackage{pythonhighlight}
\setlength{\headheight}{30pt}
\setlength{\footskip}{20pt}



\usepackage[export]{adjustbox}
\usepackage{tocloft}
\renewcommand{\cfttoctitlefont}{}
\renewcommand{\contentsname}{}
\renewcommand{\cftsecleader}{\cftdotfill{\cftdotsep}}

\setlength{\cftbeforesecskip}{0.5em}



\usepackage{hyperref}    % For hyperlinks
\usepackage{xurl}        % For better URL handling
\hypersetup{
	colorlinks=true,
	linkcolor=blue!50!black,
	urlcolor=blue,       % Color for URLs
}


\definecolor{ChineseGold}{HTML}{C59401}
\definecolor{AmericanGold}{HTML}{D3AF37}
\definecolor{MetallicSunburst}{HTML}{A77C37}
\definecolor{GoldenBrown}{HTML}{996515}
\definecolor{DarkBrown}{HTML}{674222}
\definecolor{SkyBlue}{HTML}{87CEEB}      % Soft and bright
\definecolor{BabyBlue}{HTML}{89CFF0}     % Gentle, pastel-like
\definecolor{SteelBlue}{HTML}{4682B4}    % Rich but not overpowering
\definecolor{RoyalBlue}{HTML}{4169E1}    % Strong, slightly purplish
\definecolor{MidnightBlue}{HTML}{191970} % Almost black, deep navy
\definecolor{PrussianBlue}{HTML}{003153} % Very deep blue with a classic look
\definecolor{mainblue}{HTML}{1D73BE}

\usepackage{listings}

\definecolor{codegreen}{rgb}{0,0.6,0}
\definecolor{codegray}{rgb}{0.5,0.5,0.5}
\definecolor{codepurple}{rgb}{0.58,0,0.82}
\definecolor{backcolour}{rgb}{0.95,0.95,0.92}

\lstdefinestyle{mystyle}{
	backgroundcolor=\color{white!97!gray},   
	commentstyle=\color{codegreen},
	keywordstyle=\color{purple},
	numberstyle=\tiny\color{codegray},
	stringstyle=\color{orange},
	basicstyle=\ttfamily\scriptsize,
	breakatwhitespace=false,         
	breaklines=true,                 
	captionpos=b,                    
	keepspaces=true,                 
	numbers=left,                    
	numbersep=5pt,                  
	showspaces=false,                
	showstringspaces=false,
	showtabs=false,                  
	tabsize=2
}

\lstset{style=mystyle}

%Refer to the equation as \eqref{equation}.
\usepackage{caption}  % This package allows captioning outside of a float
\usepackage[export]{adjustbox}


\usetikzlibrary{patterns}

\usetikzlibrary{patterns.meta}
\usepackage[para]{footmisc} % Example of making footnotes run together in a paragraph

\definecolor{darkgreen}{rgb}{0.0, 0.5, 0.0}

\usepackage{datetime}

\usepackage{etoolbox}

\makeatletter
\def\tagform@#1{\maketag@@@{{Eq.~#1}}} 
\makeatother

\usepackage{ifthen}
\usepackage{calc}
\usepackage{datenumber}

\usepackage{physics}
\usepackage[outline]{contour}
\usetikzlibrary{patterns,decorations.pathmorphing}
\usetikzlibrary{arrows.meta}
\tikzset{>=latex}
\contourlength{1.1pt}

\colorlet{mydarkblue}{blue!50!black}
\colorlet{myred}{red!65!black}
\colorlet{watercol}{blue!80!cyan!10!white}
\colorlet{darkwatercol}{blue!80!cyan!20!white}
\tikzstyle{piston}=[blue!50!black,top color=blue!30,bottom color=blue!50,middle color=blue!20,shading angle=0]
\tikzstyle{water}=[draw=mydarkblue,top color=watercol!90,bottom color=watercol!90!black,shading angle=5]
\tikzstyle{vertical water}=[water,
top color=watercol!90!black!90,bottom color=watercol!90!black!90,middle color=watercol!80,shading angle=90]
\def\tick#1#2{\draw[thick] (#1)++(#2:0.1) --++ (#2-180:0.2)}



\newcounter{deadlineyear}\setcounter{deadlineyear}{2025}
\newcounter{deadlinemonth}\setcounter{deadlinemonth}{4}
\newcounter{deadlineday}\setcounter{deadlineday}{9}
\newcounter{deadlinetime}\setcounter{deadlinetime}{1439} % Default: 23:59 (1439 minutes)
\newcounter{mydatenumber}
\newcounter{currentdate}
\newcounter{daysdiff}
\newcounter{currenttime}
\newcounter{totalminutes}
\newcounter{displaydays}
\newcounter{remainingmins}
\newcounter{displayhours}
\newcounter{displaymins}

% ====== MACROS ======
% Set deadline time (e.g., \setdeadlinetime{23}{59})
\newcommand{\setdeadlinetime}[2]{%
	\setcounter{deadlinetime}{#1 * 60 + #2}%
}

% Main calculation command
\newcommand{\timeUntilDeadline}{%
	% Calculate days between dates
	\setmydatenumber{mydatenumber}{\thedeadlineyear}{\thedeadlinemonth}{\thedeadlineday}%
	\setmydatenumber{currentdate}{\the\year}{\the\month}{\the\day}%
	\setcounter{daysdiff}{\themydatenumber - \thecurrentdate}%
	
	% Get current time in minutes since midnight
	\setcounter{currenttime}{\time}%
	
	% Calculate total remaining minutes (deadline time - current time)
	\setcounter{totalminutes}{\thedaysdiff * 1440 + \thedeadlinetime - \thecurrenttime}%
	
	% Check deadline status
	\ifnum\thetotalminutes < 0
	\textbf{\color{red}Deadline passed!}%
	\else
	% Calculate time components
	\setcounter{displaydays}{\thetotalminutes / 1440}%
	\setcounter{remainingmins}{\thetotalminutes - \thedisplaydays * 1440}%
	\setcounter{displayhours}{\theremainingmins / 60}%
	\setcounter{displaymins}{\theremainingmins - \thedisplayhours * 60}%
	
	% Format output
	\textbf{%
		\ifnum\thedisplaydays > 0
		\thedisplaydays\ day\ifnum\thedisplaydays > 1 s\fi%
		\ifnum\thedisplayhours > 0
		\ifnum\thedisplaymins > 0, \else\ and \fi%
		\else
		\ifnum\thedisplaymins > 0\ and \fi%
		\fi%
		\fi%
		\ifnum\thedisplayhours > 0
		\thedisplayhours\ hour\ifnum\thedisplayhours > 1 s\fi%
		\ifnum\thedisplaymins > 0\ and \fi%
		\fi%
		\ifnum\thedisplaymins > 0
		\thedisplaymins\ minute\ifnum\thedisplaymins > 1 s\fi%
		\fi%
		\ left%
	}%
	\fi
}

\definecolor{mygreen}{HTML}{07a56d}

\newtcbtheorem{briefillus}{Definiton}{
	enhanced,
	sharp corners,
	attach boxed title to top left={
		xshift=0pt, 
		yshift=-\tcboxedtitleheight, 
		yshifttext=-\tcboxedtitleheight/2-0.5em
	},
	colback=white,
	colframe=black,
	fonttitle=\bfseries,
	coltitle=white,
	boxed title style={
		rounded corners,
		arc=2pt,
		size=small,
		colback=black,
		colframe=black,
	},
	leftrule=0pt, 
	rightrule=0pt, 
}{thm}

\usepackage{titlesec}

\titleformat{\section}[block]{\vspace*{-10pt}\centering\normalfont\Large\bfseries\color{mainblue}}{\color{mainblue}\thesection}{1em}{}
\titleformat{\subsection}[block]{\normalfont\large\bfseries\color{mainblue}}{\color{mainblue}\thesubsection}{1em}{}
\titleformat{\subsubsection}[block]{\normalfont\normalsize\bfseries\color{mainblue}}{\color{mainblue}\thesubsubsection}{1em}{}

\begin{document}

\thispagestyle{empty}

\color{white}
\tikz[remember picture,overlay] \node[opacity=1,inner sep=0pt] at (current page.center){\includegraphics[width=\paperwidth,height=\paperheight]{images/A4-document-placeholder-medium-blue_0.jpg-ezgif.com-webp-to-jpg-converter(2).jpg}};

\vspace*{\fill}
\begin{center}
	\textbf{\Huge IMechE Design Challenge}\\[10pt]
	\LARGE \textbf{Group Report and Logbook}
\end{center}
\vspace*{\fill}

\Large    
\begin{tabular}{@{}l l l@{}}
	\textbf{Submitted by:}  & Samatar Ahmed (Group Leader) \phantom{ssssssssss}& K2374854\\
	& Sakariye Abiikar & K2371673 \\
	& Jayden Balgobind & K2421484\\
	& Hector Huser & K2367380\\
	& Hashaam Khan & K2371729\\
	& Josh Mossman & K2457119\\
\end{tabular}

\vspace*{\fill}

\begin{tabular}{@{}l l@{}}
	\textbf{Key Dates:} & Date of practical: Thursday 3$^{\text{rd}}$ April, 2025 \\
	& Deadline: Wednesday 9$^{\text{th}}$ April, 2025 23:59\\
	& Last Updated: \today\, \currenttime \\
	& Days left until deadline: \timeUntilDeadline \\
\end{tabular}
\vspace*{\fill}

\large
\newpage\thispagestyle{empty}\newgeometry{top=0.7in,bottom=0.6in,left=0.8in,right=0.8in}
\tikz[remember picture,overlay] \node[opacity=0.5,inner sep=0pt] at (current page.center){\includegraphics[width=\paperwidth,height=\paperheight]{images/a13d25fa5178ce400e90e65f61d696d3.jpg}};

\color{black}
\vspace*{\fill}
\noindent
\begin{tblr}{
		colspec={Q[5.4cm]Q[5.4cm]Q[5.4cm]},
		hlines,vlines,
		cells={valign=m,halign=c},
		rows={ht=2\baselineskip},
		row{1}={ht=1\baselineskip,font=\bfseries,fg=mainblue!90!white},
	}
	MODULE NO. & MODULE TITLE & MODULE LEADER \\
	EG4023 & Introduction to Engineering Design and Manufacture & Dr Andy Curley \\\hline
	Assignment Title: & \SetCell[c=2]{c} IMechE Design Challenge Project --- Group Report and Logbook &  \\	
\end{tblr}\\[1em]
\begin{tblr}{
		colspec={Q[5.4cm]Q[11.24cm]},
		hlines,vlines,
		cells={valign=m,halign=c},
		rows={ht=2\baselineskip},
	}
	Group Name: & Group 1  \\
\end{tblr}\\[1em]
\begin{minipage}{0.98\textwidth}
	\vspace*{0.1mm}
	\begin{center}
		\fbox{
			\begin{minipage}[c][16cm][c]{\textwidth}
				\centering
				\includegraphics[height=10cm]{example-image}
			\end{minipage}
		}
	\end{center}
	\vspace*{1em}
\end{minipage}
\vspace*{\fill}


\newpage\thispagestyle{empty}

\tikz[remember picture,overlay] \node[opacity=0.5,inner sep=0pt] at (current page.center){\includegraphics[width=\paperwidth,height=\paperheight]{images/a13d25fa5178ce400e90e65f61d696d3.jpg}};

\noindent\vspace*{4em}

\begin{center}
	\LARGE \textbf{\textcolor{mainblue}{Contribution Table}}\\[2em]
	\large\vspace*{2em}
\begin{tblr}{
		colspec={Q[5cm]Q[5cm]Q[5cm]},
		hlines,vlines,
		cells={valign=m,halign=c},
		rows={ht=4\baselineskip},
		row{1}={ht=1.5\baselineskip,font=\bfseries,fg=mainblue!90!white},
	}
	 Student & Contribution & Picture \\ 
	 Samatar Ahmed &  & \includegraphics[width=2cm,valign=c]{images/profile.png} \\ 
	 Sakariye Abiikar &  & \includegraphics[width=2cm,valign=c]{images/profile.png} \\ 
	 Jayden Balgobind &  &  \includegraphics[width=2cm,valign=c]{images/profile.png}\\
	 Hector Huser &  & \includegraphics[width=2cm,valign=c]{images/profile.png} \\ 
	 Hashaam Khan &  & \includegraphics[width=2cm,valign=c]{images/profile.png}\\
	 Josh Mossman &  & \includegraphics[width=2cm,valign=c]{images/profile.png} \\
\end{tblr}
\end{center}	
\vspace*{\fill}

\normalsize
\newpage\newgeometry{top=0.2in,bottom=1in,left=0.8in,right=0.8in}\tikz[remember picture,overlay] \node[opacity=0.5,inner sep=0pt] at (current page.center){\includegraphics[width=\paperwidth,height=\paperheight]{images/a13d25fa5178ce400e90e65f61d696d3.jpg}};
\noindent\vspace{7em}\pagenumbering{gobble}
\begin{center}
	\LARGE \textbf{\textcolor{mainblue}{Table of Contents}}\\[-7em]
\end{center}
{
	\hypersetup{linkcolor=black}
	\tableofcontents
}    


\large\newpage\restoregeometry
\noindent\pagenumbering{arabic}\setcounter{page}{1}

\section{Abstract}
This report/logbook serves as a comprehensive account of our group's involvement in the IMechE Design Challenge 2025, which tasks first-year engineering students with designing, building, and testing an autonomous robotic charging device. Although the design appeared simple at first, transforming our ideas into a working prototype presented significant challenges. Throughout the project, we encountered technical and mechanical obstacles that required creative problem-solving and adaptation. This report details the design process, challenges faced, solutions implemented, and the collaborative efforts that led to the final product. Through this challenge, we gained invaluable experience in practical engineering, teamwork, and the application of analogue electronics principles, all while exploring the potential of autonomous and sustainable technologies.

\newpage
\section{Introduction}
The \textit{Design Challenge}, organized annually by the Institution of Mechanical Engineers (IMechE), invites first-year undergraduate engineering students to participate in a competition that tests their skills in designing, building, and testing innovative solutions. This year, the challenge involves designing an \textbf{autonomous robotic charging device}. 
\subsection{Competition Overview}
Design a self-contained device that autonomously connects/disconnects from a charging target, simulating EV charging. It must navigate a 1m-wide horizontal track (1.4-4.0m length) to engage a vertical wall (0.3m min height) under these rules:
\subsubsection*{Key Requirements}
\begin{itemize}[itemsep=-0.5mm,topsep=0pt]
	\item Complete mission autonomously within 3 minutes
	\item Tolerate real-world conditions (8mm/1.5m level variance, 3mm surface gaps)
	\item Engage charging target (50mm height for single, 50/150mm for double target)
	\item Maintain contact for specified charging duration
	\item Handle three distance ranges:
	\begin{itemize}[noitemsep,topsep=0pt]
		\item Short: 1.4-2.2m
		\item Medium: 2.4-3.0m
		\item Long: 3.2-4.0m
	\end{itemize}
\end{itemize}
\subsubsection*{Track Specifications}
\begin{itemize}[noitemsep,topsep=0pt]
	\item \textbf{Material}: Commercial wood boards (plywood/OSB/MDF) or approved hard floors
	\item \textbf{Construction}: Joined boards (3mm max level variance, 2mm max gaps)
	\item \textbf{Layout}: Multiple lanes with 1m spacing
\end{itemize}
\begin{center}
	\includegraphics[width=0.8\textwidth]{extracted_images/image_4_2.png}\\
	\small\textbf{Figure 3:} Lane setup (dimensions in mm)
\end{center}
\subsubsection*{Measurement Tolerance}
$\pm$10mm using a single tape measure. Barrier positions vary in 50mm increments within each range.


\section{Product Design Specification}
Info is catered to the Foundation Group.
\subsection{Dimensional Requirements}
\begin{itemize}[itemsep=-0.7mm]
	\item Maximum working envelope: 400mm $\times$ 400mm $\times$ 400mm (including all components)
	\item Must maintain envelope compliance during all competition movements
	\item Front plunger (plug simulator):
	\begin{itemize}[noitemsep,topsep=0pt]
		\item Max dimensions: 10mm $\times$ 10mm
		\item Minimum protrusion: 10mm from device body when disengaged
	\end{itemize}
\end{itemize}

\subsection{Required Components}
\begin{itemize}[itemsep=-0.7mm]
	\item \textbf{Rear datum pointer} (RS Components 397-4954):
	\begin{itemize}[noitemsep,topsep=0pt]
		\item Must remain vertical (pointing downward)
		\item Max 6mm clearance from track surface
		\item Determines device positioning
	\end{itemize}
	\item \textbf{Single front plunger} (non-adjustable height for Foundation Group)
\end{itemize}

\subsection{Visual Indicators}
\begin{itemize}[noitemsep,topsep=0pt]
	\item Continuous green light during operation
	\item Red light + audible signal upon successful wall engagement
\end{itemize}

\subsection{Technical Restrictions}
\begin{itemize}[itemsep=-0.7mm,topsep=0pt]
	\item \textbf{No programmable circuitry} permitted (analog systems only)
	\item All repairs must be completed within allocated competition time
	\item Device must be fully self-contained (no external control)
\end{itemize}

\begin{figure}[h]
	\centering
	\begin{minipage}{0.45\textwidth}
		\centering
		\includegraphics[width=0.8\linewidth]{extracted_images/image_10_2.png}
		\caption{Front plunger requirements}
	\end{minipage}
	\begin{minipage}{0.45\textwidth}
		\centering
		\includegraphics[width=0.8\textwidth,height=3.2cm]{extracted_images/image_10_1.png}
		\caption{Datum pointer specification}
	\end{minipage}
\end{figure}
\subsection*{Key Differences from Advanced Category}
\begin{itemize}[itemsep=-0.7mm]
	\item No height adjustment capability required
	\item No programming or digital components allowed
	\item Simplified single-target operation
\end{itemize}


\newpage\newgeometry{left=0.8in,right=0.8in,top=1in,bottom=0.69in}
\section{Design Concepts}

\begin{minipage}{0.59\textwidth}
\subsection{Design Concept 1}
This was our initial concept idea, which had the advantage of being simple with minimal refinement. However, its main drawback was the high cost. While the model was accurate and met the specifications, its complexity and expensive manufacturing process made it difficult to produce.\\[8pt]
A major issue with this design is that high manufacturing costs can make large-scale production unfeasible. When a product is too costly to produce, it affects everything, from labor and machinery costs to production time, resulting in increased expenses and longer production timelines.
\end{minipage}\hfill
\begin{minipage}{0.4\textwidth}
\centering
\includegraphics[width=1\textwidth]{images/image_6_2-Photoroom.png}
\captionof{figure}{Design Concept 1}
\end{minipage}\\[8pt]
The positives of this design is that it does meet the required specifications as it does fit the purpose of its role.\\[1em]
\begin{minipage}{0.59\textwidth}
	\subsection{Design Concept 2}
	This was our second design, which was simple and aesthetically pleasing. Its major strength was its ease of construction and compact size, which provided extra space for potential part exchanges or swaps. It was an adaptable design, but the main drawback was its susceptibility to breakages.\\[8pt]
	The negatives of this build are that it is prone to breakages, making it unreliable. This is crucial because during the testing phase, the design must pass to function correctly. If it is prone to breaking easily, it will reduce its longevity as materials deteriorate over time. The small size also means less room for improvements if needed.
\end{minipage}\hfill
\begin{minipage}{0.35\textwidth}
	\centering
	\includegraphics[width=1\textwidth]{images/image_7_2-Photoroom.png}
	\captionof{figure}{Design Concept 2}
\end{minipage}\\[4pt]
However, it is an aesthetic and versatile design, easy to swap parts if it breaks. The compactness is efficient for aerodynamics, and the lightness comes from the absence of bulky components.\\[1em]
\begin{minipage}{0.59\textwidth}
	\subsection{Design Concept 3}
	This was our final concept idea, which, like the first, had minimal refinement. Its major drawback was the cost, as it involved many unique moving parts. Although this design was durable and strong, it exceeded the budget outlined in the technical specifications.\\[8pt]
	The overall budget of this design was very high due to the numerous parts required, which increased manufacturing costs. There was also a lack of refinement, leading to inefficiencies in the design itself. The complexity of the build made assembly difficult, resulting in a higher likelihood of errors that could have been easily fixed with a simpler design. 
\end{minipage}\hfill
\begin{minipage}{0.4\textwidth}
	\centering
	\includegraphics[width=1\textwidth]{images/image_8_2-Photoroom.png}
	\captionof{figure}{Design Concept 3}
\end{minipage}\\[8pt]
Despite this, it was a durable and strong design that could withstand impact and perform well.

\newpage\restoregeometry
\section{CAD}


\newpage\newgeometry{left=0.8in,right=0.8in,top=1in,bottom=0.69in}
\section{Electronics}
Given the scope of this report and its intended audience, this section provides a overview of my electronics development journey---from initial concept to final implementation. Rather than a deep \textbf{technical} dive, it serves as a reflective summary, documenting key learnings, challenges, and methodological insights.\\[8pt]
At this academic level, the expectations for hands-on electronics proficiency were limited, and formal guidance was sparse. While I acknowledge that instructors prioritized foundational theory over practical support, this approach posed difficulties for beginners. Mastering the necessary electronics concepts---especially under time constraints---can take months, and the lack of structured mentorship made the process more daunting.\\[8pt]
The primary advice given was to use \textit{Tinkercad} for construction. While this tool has merits, I found it insufficient for my needs. Instead, I adopted a hybrid approach, blending theoretical understanding with targeted research to bridge knowledge gaps.


\subsection{Starting Structure}
My process started with building the core of the system: a \textbf{MOSFET-based H-bridge} using complementary P-channel and N-channel MOSFETs. At this stage, the focus was on creating two simple and clear signal inputs to control the motor: \textbf{DIR} for direction (CW/CCW) and \textbf{PWR} for enabling or disabling the motor. I approached the design with basic logic gates, using {AND} and {NOT} gates to process the inputs into the necessary control signals for the MOSFETs. The logic was straightforward yet crucial, as it needed to handle some key responsibilities:
\begin{enumerate}[itemsep=-1mm]
	\item When {PWR = 0}, the system would disable the motor, allowing it to coast to a stop.
	\item When {PWR = 1}, the {DIR} signal would control which MOSFET pairs were activated, determining the motor’s direction.
\end{enumerate}
This initial switching matrix became the backbone of the system, and everything I built afterward had to integrate seamlessly with this simple but essential layer. I made sure to carefully test and validate the gate network, ensuring it was stable and reliable before moving on to more complex features. Ultimately, the success of the entire system depended on the reliability of this basic setup.

\subsection{Prioritizing Pause Logic}
With the H-bridge logic in place, the next challenge was to implement a method to {pause the motor} for a fixed duration, triggered via physical contact. I decided to use a \textbf{555 timer} with a push-button trigger. The timer was configured to generate a $\sim$15-second HIGH pulse upon activation, with the configuration $$1.1 \times 470\ \mu{F} \times 27\ {k}\Omega\approx 15s$$
The HIGH output from the timer was {inverted} and used to control one of the input pins for an AND gate ({PWR}). This way, while the timer was active, the motor would {coast/stop}. Once the timer expired, the motor would resume operation. By this point, the directional control was fixed, and I planned to tackle that aspect later, after establishing a reliable pause-reset flow.

\subsection{Direction Logic}
\subsubsection{Button-Based Reversal}
With the timer working, I wanted to make the motor {switch direction} after each cycle. My first idea was to {reuse the same push-button} that triggered the 555 timer to also reverse the direction. I designed a separate logic circuit using transistors, capacitors, and button presses to toggle the signal. It worked perfectly in isolation. However, when I tried to integrate it into the 555 setup, things didn’t go as planned. The reversal circuit and the timer had different assumptions about how the button behaved, leading to conflicting logic and erratic behavior. I ultimately abandoned this approach.

\subsubsection{Relay Logic}
After abandoning the button-based method, I came up with a better idea: to use the {555’s own output pulse} to handle direction switching. This would separate the trigger from the direction logic, making the system more modular.\\[8pt]
In {CircuitJS}, I built a setup using a \textbf{relay} that flipped the direction signal each time the 555 pulse occurred. It worked as expected-the relay acted like a physical toggle, and I used invisible labels to manage virtual connections and logic states.\\[8pt]
However, when I tried to replicate this in {Tinkercad}, it didn’t work. Tinkercad couldn’t support the internal relay wiring or virtual labels I had used in the simulation, so I had to rethink the approach. After much trial and error, I found it difficult to make the setup work in that platform-it just felt bulky and inefficient. I needed something cleaner and more logical, not forced.

\subsubsection{T Flip-Flop}
Frustrated, I turned to an online forum. A helpful suggestion came through: {use a T flip-flop.}\\
This turned out to be exactly what I needed. A \textbf{T flip-flop} toggles its state on every clock pulse-in my case, the falling edge of the 555 timer. This solution completely decoupled the direction logic from the button and worked perfectly both in theory and in practice.\\[3pt]
I set it up so that:
\begin{itemize}[itemsep=-1mm]
	\item The {555 timer} would disable the motor for 15 seconds.
	\item When the pulse ended, the falling edge would clock the {T flip-flop}, toggling the direction.
	\item The T flip-flop output would then feed into the direction input of the H-bridge.
\end{itemize}
This setup was stable, minimal, and, most importantly, felt right.

\subsection{Breadboarding}
Moving from simulation to real hardware brought its own set of challenges.
\begin{itemize}[itemsep=-1mm]
	\item First, the {T flip-flop} (actually implemented using a J-K flip-flop with both J and K set high to mimic T flip-flop behavior, and using {PRE} and {CLR} as reset) behaved unpredictably. It turned out the trigger was {floating}, and I hadn’t added pull-down resistors (new to me at the time). Simulations hadn’t shown this, but adding the resistors fixed the issue.
	\item I also realized the 555’s button needed debouncing. Since the 555 timer was in monostable mode, if the button was pressed longer than the pulse duration, it wouldn’t change state as expected. To solve this, I added two 10kΩ resistors and a 0.1µF capacitor with the button. This ensured the 555 timer was triggered only once per button press, even if the button was held down longer than the pulse duration.
	\item By now, I had long since moved beyond Tinkercad. It couldn’t simulate well and too many things didn’t behave as they did in the real world.
\end{itemize}
\subsection{Final Behavior -- What It Actually Does}
Here’s what the final circuit accomplishes:
\begin{itemize}[itemsep=-1mm]
	\item Pressing a {button} triggers a 15-second pulse via a {555 timer}.
	\item While the pulse is active, the {inverted output} disables the motor (via the H-bridge's enable line).
	\item When the timer ends, the {T flip-flop toggles} the direction.
	\item Logic gates maintain signal integrity and enforce state rules.
	\item A {limit switch} cuts the circuit entirely if triggered, providing a safety cutoff.
\end{itemize}
Additional tasks such as cleaning signals, filtering, and ensuring smooth overall behavior were then implemented, along with essential components like the buzzer and fuse box, placed in their respective locations.\\[8pt]
While I had intended to discuss the PCB creation, that step was never completed---thanks to the university's machine breaking down just before I could proceed.



\newpage

\subsection{Circuit Diagrams and Breadboard Setup}
The underlying main circuit schematic is shown below:\vspace{-1em}
\begin{figure}[H]
	\centering
	\includegraphics[width=0.4\textwidth]{images/swappy-20250406-185116.png}\vspace{-1em}
	\caption{Main Circuit Schematic}
\end{figure}
The breadboard schematic is presented here:\vspace{-1em}
\begin{figure}[H]
	\centering
	\includegraphics[width=0.8\textwidth]{images/Schematic_SAKARIYEPCB.sch_2025-04-06_page-0001.jpg}\vspace{-1em}
	\caption{Breadboard Schematic}
\end{figure}
The actual breadboard setup is shown below:
\begin{figure}[H]
	\centering
	\includegraphics[width=0.5\textwidth]{images/Image(2).jpg}
	\caption{Breadboard Setup}
\end{figure}\noindent
These diagrams and setups were subject to change as the project progressed. They are included here to illustrate the foundational premise of the design. You may notice discrepancies, such as the RC debounce circuit, fuse box, limit-switch, and other minor details, which were refined and adjusted throughout the process.

\newpage\restoregeometry


\section{BOM (Bill of materials)}
Parts listed for the device are stated below. Every part that was mentioned was used within the final design of the device.
\begin{itemize}[itemsep=-1mm]
	\item Wheels x 4
	\item Lead thread hub x 4
	\item Base x 1
	\item Screw and lead x 1
	\item Plastic tube x 1
	\item Thread screw x 1
	\item Motor x 1
	\item Datum x 1
	\item Belt x 1
\end{itemize}\noindent
Below is a list of the electrical components that were used within the device. Everything mentioned below was used in the final design of the device.
\begin{itemize}[itemsep=-1mm]
	\item \textbf{Chips:} SN7404N (NOT/INVERTER), SN74LS08N (AND), SN7476N (J\&K), 555 Timer
	\item \textbf{Resistors:} 33k, 27k, 10k (x2), 220 (x2)
	\item \textbf{Capacitors:} 470~\textmu F, 100~nF
	\item \textbf{LEDs:} Green, Red
	\item \textbf{MOSFETs:} 2 NPN, 2 PNP
	\item \textbf{Switches:} Main switch (SPDT), Limit switch, Push switch
	\item \textbf{Source:} 6V (4 batteries)
	\item \textbf{Motor:} Unbranded/Generic (50 and 100 rpm testing)
	\item \textbf{Miscellaneous:} Buzzer, Fusebox, Wires, Heat Shrink Insulation Tape
\end{itemize}
These were parts that were essential when building the device. Since we used all components, we did not waste any money, which could have impacted our costs — we could have used the money to purchase better equipment or parts if needed.

\newpage
\section{Manufacturing process}


\newpage
\section{Final design}
\begin{center}
	\includegraphics[width=0.7\textwidth]{extracted_images2/image_11_2.png}
\end{center}\vspace{0.6em}\noindent
As you can see from the picture provided of the car, you can see that we have fully manufactured a working car that follows all the guidelines and specifications that were required. The main aim of this competition was to create a device that could reach a target there and back under three minutes. The datum pointer should start in the centre of the target and move towards the wall. It should then touch the end of the wall and stay there for 15 seconds then return to the same position as before where the datum pointer started from. The device was safe to use since we ensures that all rules were carefully followed such as electrical connections not being exposed or having the car not fit within the maximum envelope.\\[1em]
As you can see, we received no penalties given since our datum pointer was not too high from the ground. Our front plunger was not larger that it was allowed, and it could realign between the two vertical targets. If the above was not completed it would have resulted in penalties for our team.\\[1em]
The size of envelope is 400x400x400mm which our device was within regulation of. The device should also be able to fit within the envelope throughout the whole competition. We ensured that the datum pointer is in a vertical position and is pointing downwards.

\newpage
\section{References}

\end{document}